% This is samplepaper.tex, a sample chapter demonstrating the
% LLNCS macro package for Springer Computer Science proceedings;
% Version 2.20 of 2017/10/04
%
\documentclass[runningheads]{llncs}
%
\usepackage{amsmath}
\usepackage{booktabs} % For pretty tables
\usepackage{caption} % For caption spacing
\usepackage{subcaption} % For sub-figures
\usepackage{graphicx}
\usepackage{pgfplots}
\usepackage[all]{nowidow}
\usepackage[utf8]{inputenc}
\usepackage[margin=1in]{geometry}
\usepackage{tikz}
\usetikzlibrary{er,positioning,bayesnet}
\usepackage{multicol}
\usepackage{algpseudocode,algorithm,algorithmicx}
\usepackage{minted}
\usepackage{hyperref}
\usepackage[inline]{enumitem} % Horizontal lists
% Used for displaying a sample figure. If possible, figure files should
% be included in EPS format.
%
% If you use the hyperref package, please uncomment the following line
% to display URLs in blue roman font according to Springer's eBook style:
% \renewcommand\UrlFont{\color{blue}\rmfamily}

\newcommand{\card}[1]{\left\vert{#1}\right\vert}
\newcommand*\Let[2]{\State #1 $\gets$ #2}
\definecolor{blue}{HTML}{1F77B4}
\definecolor{orange}{HTML}{FF7F0E}
\definecolor{green}{HTML}{2CA02C}

\pgfplotsset{compat=1.14}

\renewcommand{\topfraction}{0.85}
\renewcommand{\bottomfraction}{0.85}
\renewcommand{\textfraction}{0.15}
\renewcommand{\floatpagefraction}{0.8}
\renewcommand{\textfraction}{0.1}
\setlength{\floatsep}{3pt plus 1pt minus 1pt}
\setlength{\textfloatsep}{3pt plus 1pt minus 1pt}
\setlength{\intextsep}{3pt plus 1pt minus 1pt}
\setlength{\abovecaptionskip}{2pt plus 1pt minus 1pt}

\begin{document}

\title{AER303 Aerospace Laboratory - Aerodynamic Forces on an Airfoil}
%\titlerunning{Add subtitle}

\author{Eric Dai\inst{1} \and Jai Willems\inst{2} \and Mingde Yin\inst{3}}
%\authorrunning{F. Author et al.}

\institute{Division of Engineering Science, University of Toronto, Toronto, Canada \email{eric.dai@mail.utoronto.ca}\\ \and Division of Engineering Science, University of Toronto, Toronto, Canada \email{jai.willems@mail.utoronto.ca}\\ \and Division of Engineering Science, University of Toronto, Toronto, Canada\\ \email{mingde.yin@mail.utoronto.ca}}

\maketitle


% -----------------------------------------------------------------------------
%   Abstract
% -----------------------------------------------------------------------------


\begin{abstract}

A Clark Y airfoil is characterized at varying angles of attack in terms of $C_L$, $C_D$, and $C_M$ from surface pressure and wake velocity measurements in a subsonic wind tunnel. Measurements are collected using two separate methods, and their comparative effectiveness is evaluated.

\keywords{Airfoil \and Stall \and Inclined Manometer \and Scanivalve}
\end{abstract}


% -----------------------------------------------------------------------------
%   Nomenclature
% -----------------------------------------------------------------------------


\section*{Nomenclature}


% -----------------------------------------------------------------------------
%   Introduction
% -----------------------------------------------------------------------------


\section{Introduction}
Airfoils are two-dimensional cross sections of wings. They produce lift by redirecting flow over their surface such that the resulting surface pressure distribution causes a net upward force. At the same time, skin friction effects impact viscous drag forces on the airfoil, and a combination of both impart a pitching moment on the airfoil. The pressure distribution over an airfoil does not remain constant over different angles of attack, and changes sharply near stall.\\
\\
Stall is characterized by a significant separation in flow from the airfoil, resulting in loss of lift, increased drag, and a positive pitching moment. Stall leads to unpredictable behaviour of an airfoil, and loss of control in an aircraft. It is important to characterize the effect of a varying angle of attack on the above performance parameters of an airfoil.\\
\\
This report describes the experimental performance characterization of a Clark Y airfoil at varying angles of attack. Measurements of the surface pressure distribution of the airfoil, along with velocity measurements of the wake are used to determine the lift, moment, and drag coefficients on the airfoil up to the point of stall. Measurements are performed using two separate methods, and their comparative effectiveness is evaluated.

% -----------------------------------------------------------------------------
%   Experimental Set-Up
% -----------------------------------------------------------------------------


\section{Experimental Set-Up}
\subsection{Procedure}
\subsection{Apparatus}
\begin{figure}
    \centering
    \includegraphics{}
    \caption{Caption}
    \label{fig:my_label}
\end{figure}

% -----------------------------------------------------------------------------
%   Results and Discussion
% -----------------------------------------------------------------------------


\section{Results and Discussion}


% -----------------------------------------------------------------------------
%   Conclusion
% -----------------------------------------------------------------------------


\section{Conclusion}


% -----------------------------------------------------------------------------
%   Bibliography
% -----------------------------------------------------------------------------


\bibliographystyle{ieeetr}
\bibliography{biblio}


% -----------------------------------------------------------------------------
%   Appendix
% -----------------------------------------------------------------------------


\appendix
\section{Pressure Measurements}


\end{document}
